\documentclass[titlepage=false,12pt]{scrreprt}

\usepackage[utf8]{inputenc}
\usepackage[T1]{fontenc}
\usepackage{lmodern}
\usepackage[ngerman]{babel}
\usepackage{amsmath}
\usepackage[backend=bibtex]{biblatex}
\usepackage{tikz}
\usepackage{amsthm}
\usepackage{amssymb}
\usepackage{xcolor}
\usepackage{framed}
\usepackage{tabto}
\usepackage{capt-of}
\usepackage{pgfplots} % loads tikz
\pgfplotsset{compat=1.7}
\usepackage{csquotes}
\usepgfplotslibrary{fillbetween}
\usetikzlibrary{intersections}

\pgfdeclarelayer{bg}
\pgfsetlayers{bg,main}

\colorlet{shadecolor}{gray!15}

\usepackage{hyperref}
\hypersetup{
	colorlinks,
	citecolor=black,
	filecolor=black,
	linkcolor=black,
	urlcolor=black
}

\newtheorem{definition}{Definition}[chapter]
\AfterEndEnvironment{definition}{\noindent\ignorespaces}

\newcommand{\listingsettings}{%
	\lstset{%
		language=C++,			% Standardsprache des Quellcodes
		%numbers=left,			% Zeilennummern links
		%stepnumber=1,			% Jede Zeile nummerieren.
		%numbersep=5pt,			% 5pt Abstand zum Quellcode
		%numberstyle=\tiny,		% Zeichengrösse 'tiny' für die Nummern.
		breaklines=true,		% Zeilen umbrechen wenn notwendig.
		breakautoindent=true,	% Nach dem Zeilenumbruch Zeile einrücken.
		postbreak=\space,		% Bei Leerzeichen umbrechen.
		tabsize=2,				% Tabulatorgrösse 2
		basicstyle=\ttfamily\footnotesize, % Nichtproportionale Schrift, klein für den Quellcode
		showspaces=false,		% Leerzeichen nicht anzeigen.
		showstringspaces=false,	% Leerzeichen auch in Strings ('') nicht anzeigen.
		extendedchars=true,		% Alle Zeichen vom Latin1 Zeichensatz anzeigen.
		captionpos=b,			% sets the caption-position to bottom
		%backgroundcolor=\color{ListingBackground}, % Hintergrundfarbe des Quellcodes setzen.
		xleftmargin=0pt,		% Rand links
		xrightmargin=0pt,		% Rand rechts
		frame=single,			% Rahmen an
		frameround=ffff,
		rulecolor=\color{darkgray},	% Rahmenfarbe
		%fillcolor=\color{ListingBackground},
		keywordstyle=\color[rgb]{0.133,0.133,0.6},
		commentstyle=\color[rgb]{0.133,0.545,0.133},
		stringstyle=\color[rgb]{0.627,0.126,0.941},
		aboveskip=1.5em,
	}
}

\clubpenalty = 10000 % schließt Schusterjungen aus (Seitenumbruch nach der ersten Zeile eines neuen Absatzes)
\widowpenalty = 10000 % schließt Hurenkinder aus (die letzte Zeile eines Absatzes steht auf einer neuen Seite)
\displaywidowpenalty=10000

\bibliography{bibliography}

\title{MVVM: Model-View-ViewModel}
\subtitle{Advanced Softwareengineering - DHBW Stuttgart}
\author{Nico Vogel, Lukas Sopora}
\date{31.12.2019}

\begin{document}
	\maketitle
	{\renewcommand\clearpage\relax
		\tableofcontents}
	\newpage


	\chapter{Was ist das Problem das MVVM angeht?}
	\begin{itemize}
		\item ui fix couplled mit logik
		\item schwierig einzelne komponenten oder ganze ui auszutauschen
		\item cross platform nicht möglich
		\item wiederverwendung der logik nicht möglich
	\end{itemize}


	\chapter{Beschreibung der Komponenten von MVVM}
	\begin{itemize}
		\item Komponenten
		\begin{itemize}
			\item Model
			\begin{itemize}
				\item Enitiät
				\item enthält nur daten oder logik für sich selbst
				\item Programmlogik beschränkt sich auf Validierung
				% beispiel einbringen "CollectionManager"
				% daten: irgendwelche objekte
				% logik: filter um nur bestimmte objekte der collection zurückzugeben
			\end{itemize}
			\item ViewModel
			\begin{itemize}
				\item business logik
				\item verwendet beliegig viele models
				\item stellt werte bereit für die ui
				\item ließt werte ein aus der ui
				\item leitet events weiter an die logik
				\item kennt Model
			\end{itemize}
			\item View
			\begin{itemize}
				\item nur design
				\item eigenständig
				\item kennt ViewModel 
			\end{itemize}
			\item Übersicht komponenten
			% bild wie die komponenten miteinander agieren
			% Model gibt datenstruktur vor
			% ViewModel gibt daten für eine view vor 
			\item Übersicht einteilund in das Application Layered model
			% soll zeigen das die ViewModel presentation und Business zugleich sind.
		\end{itemize}
		\item Kommunikation
		\begin{itemize}
			\item OneWay Binding: von View zu ViewModel oder von ViewModel zu View
			\item TwoWay Binding: Von der View zum ViewModel und gleichzeitig auch andersrum
			% egal wers aktualisiert, der wert wird an den anderen weiter gegeben
			\item Events: die view reicht ein event an das ViewModel weiter 
			% methoden aufruf in dem ViewModel
		\end{itemize}
	\end{itemize}


	\chapter{Beispiel des MVVM Patterns in C\# WPF}


	\chapter{Vergleich zu dem MVC Pattern}


	\chapter{Vergleich zu dem MVP Pattern}


	\chapter{Kritische betrachtung des MVVM Patterns}
\end{document}